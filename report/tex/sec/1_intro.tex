\section{Introduction}
Football has become the universal language, transcending ethnicities and age groups worldwide. Its growing global popularity has significantly increased the commercial value of the sport’s key figures—the players. Over the past decade, the football transfer market, where clubs pay fees to acquire players from other teams, has surged to nearly \$10 billion. This monumental market attracts immense global attention, making news about potential player moves, commonly referred to as transfer rumors, a focal point of discussion. These rumors influence fans, analysts, and even club strategies, highlighting their substantial impact on the football ecosystem.

However, transfer rumors exhibit considerable variability in accuracy and reliability. Some sources are renowned for providing near-confirmatory updates with phrases like "here we go," while others often announce unexpected breaking news that ultimately proves unfounded. This inconsistency underscores the necessity for a systematic analysis of the patterns and predictors that determine the reliability of these rumors. Understanding these dynamics can help stakeholders navigate the complex landscape of football transfers with greater insight and confidence.

To address this need, we utilized the comprehensive database \href{www.transfermarkt.co.uk}{Transfermarkt}, a highly reputable source known for its extensive archive of transfer records and associated rumors contributed by football fans worldwide. We scraped 13,251 transfer rumors related to footballers who transferred between 2005 and 2023, using their actual transfer records as ground truth for validation. Recognizing that different news sources may employ distinct strategies tailored to their specific audiences and objectives, we clustered these sources based on metrics such as coverage, accuracy, and a novel “EarlyBird” metric. The EarlyBird metric quantifies the lead time before a transfer materializes, allowing us to identify common strategies and characteristics among the sources.

Furthermore, we extracted key features from the rumors and developed a predictive model to assess the likelihood of a reported transfer occurring. Our model achieved an accuracy of 73.77\%, demonstrating the effectiveness of our approach and providing valuable insights into the dynamics of football transfer reporting. This study not only sheds light on the reliability of various news sources but also offers a methodological framework for analyzing and predicting the outcomes of transfer rumors.

The remainder of this paper is structured as follows:  \autoref{sec:2_related} reviews related literature on football transfer analysis and rumor credibility; \autoref{sec:3_method} details our data collection, pre-processing, clustering techniques, and predictive modeling approach; \autoref{sec:4_exp} presents our experimental setup and results, while \autoref{sec:5_conclu} concludes with a discussion of our findings, limitations, and suggestions for future research.

